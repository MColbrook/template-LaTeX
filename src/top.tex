%%%%%%%%%%%%%%%%%%%%%%%%%%%%%%%%%%%%%%%%%%%%
% You can use 
% \verbatiminput{<file_name>} 
% to show source codes. Alternatively, use 
% \verb+<your_code_here>+ to show code inline
% or use
% \begin{verbatim}
%    ...
% \end{verbatim}
%%%%%%%%%%%%%%%%%%%%%%%%%%%%%%%%%%%%%%%%%%%%
% ../include/code_params has a customized
% code styling (for Verilog), please, modify
% to use with any other source code.
% Also, ../include/custom_comms has necessary
% functions for the lstlistings
%%%%%%%%%%%%%%%%%%%%%%%%%%%%%%%%%%%%%%%%%%%%

%\documentclass[12pt,DIV12]{scrartcl}	% I find this document type the nicest, can use beavel
\documentclass[12pt]{article} % This document is the most neutral

% ==========================================
% Modify these on your own risk... :)
\input{ ../include/include_packs}	% Packages
\input{ ../include/page_params} 	% Page layout and parameters
\input{ ../include/custom_comms}	% Custom commands
\input{ ../include/code_params} 	% Code params for LISTINGS package


% ==========================================
% Document parameters
% You can include pictures here
\addHeader{ADI}{\LaTeX~Template} % {Left}{Right}

% ==========================================
% Document information:
\title{Template for~\LaTeX~files}
% \subtitle{Tutorial}
\date{2013}
\author{ 
  Analog Devices, Inc\thanks{email \href{mailto:Zafar.Takhirov@Analog.com}{Zafar.Takhirov@Analog.com} 
    for the source codes}
}

% ==========================================
% Main Body:
\begin{document}
        \maketitle
        %\thispagestyle{empty}
        %\setcounter{page}{0}
	%\input{../src/document_formatting.tex}
        
        \section *{Motivation}
        \label{sec:motivation}
        I write a lot of documents, so why not having a template and making all of them look the same?
        
        \pagebreak
        \tableofcontents
        
        \pagebreak
        \section {Writing codes in the \LaTeX~environment}
        You can write your codes using \verb+lstlisting+. The default style is set to 
        \verb+custom_verilog+, which is a Verilog environment (defined in the 
        \verb+./include/code_params.tex+). To use the default environement use:

        \begin{lstlisting}[language={[LaTeX]{TeX}},numbers=none]
\begin{^\verb+lstlisting+^}[label=code:my_label,caption={My Caption}]
  <Some Code Here>;
  <More Code Here>;
  <etc.>
\end{^\verb+lstlisting+^}
        \end{lstlisting}
        \verb+label+ keyword allows you to reference your code, while \verb+caption+ creates a
        caption (Duh!).

        \section {Special functions that you can use}
        \label{sec:functions}
        I have written some \LaTeX functions that you can use. All of them are defined in the
        \verb+./include/+ \verb+custom_comms.tex+ file. The possible functions are:
        \begin{lstlisting}[language={[LaTeX]{TeX}},numbers=none]
\makespace % Creates spacing after the \wrapfigure
        \end{lstlisting}
        
        \subsection {Functions for collaboration}
        \label{sec:functions:collaboration}
        These functions might help you if you are writing a document with someone:
        \begin{lstlisting}[language={[LaTeX]{TeX}},numbers=none]
\fixme{This needs to be fixed! Are you sure this is right?}
        \end{lstlisting}
        Result: \\
        \fixme{This needs to be fixed! Are you sure this is right?}

        \begin{lstlisting}[language={[LaTeX]{TeX}},numbers=none]
I have written a long text, and I am not sure if whatever after that is important.
\ignore{This text although was written will not be shown on the final document.}
I have ignored it for now, to see how it would look like without it.
        \end{lstlisting}
        Result: \\
        I have written a long text, and I am not sure if whatever after that is important.
        \ignore{This text although was written will not be shown on the final document.}
        I have ignored it for now, to see how it would look like without it.

        
        \section {Included files}
        \label{sec:files}
        
        \subsection {Codes}
        \label{sec:files:codes}
        The sample codes are saved in the \verb+./codes/+ directory. Currently there is 
        a sample code called \verb+sample.sv+:
        \lstinputlisting[
          style=custom_verilog,
          label=code:sample,
          caption={This is just a sample code}
        ]{../codes/sample.sv}

        \subsection {Images}
        \label{sec:files:images}
        All the images should be stored in the \verb+./img/+ directory. Currently it is
        empty.

        \subsection {Formatting file}
        \label{sec:files:formatting}
        The \verb+src+ directory has a formatting reference. Noone follows it anyway:
        \input{../src/document_formatting.tex}

        \subsection {Bibliography}
        \label{sec:files:bibliography}
        The bibliography file is stored in the \verb+./src/+ directory. Currently has only
        a sample reference to \cite{test}.
        
        % Add bibliography
        %\pagebreak
        \bibliographystyle{unsrt}
        \bibliography{../src/top}
        
\end{document}
